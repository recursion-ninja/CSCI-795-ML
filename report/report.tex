% Options for packages loaded elsewhere
\PassOptionsToPackage{unicode}{hyperref}
\PassOptionsToPackage{hyphens}{url}

\documentclass[12pt]{diazessay}

\usepackage{amsfonts,amsmath,amssymb}
\usepackage{bashful}
\usepackage{calc}
\usepackage{etoolbox}
\usepackage[T1]{fontenc}
\usepackage{hyperref}
\usepackage[utf8]{inputenc}
\usepackage{longtable}
\usepackage[]{microtype}
\usepackage{parskip}
\usepackage{textcomp}
\usepackage{upquote}
\usepackage{xcolor}
\usepackage{xurl}

%\makesavenoteenv{longtable}
\UseMicrotypeSet[protrusion]{basicmath}
\providecommand{\tightlist}{%
  \setlength{\itemsep}{0pt}\setlength{\parskip}{0pt}}
\setcounter{secnumdepth}{-\maxdimen}

%--------------------
%	TITLE SECTION
%--------------------
%\vspace*{-2.25cm}
\title{\texttt{\LARGE{Classifying Creature Combat Capability in \\Dungeons and Dragons 5th Edition} \\\vspace{1em} {\large A Hunter College CSCI-795 Project Proposal}}} % Title and subtitle

\author{\texttt{{\Huge Team:}\\\vspace*{-0.5em}
		John Lee \\\vspace*{-0.25em}
		Alex Washburn}} % Author and institution

\date{\texttt{\today}} % Date, use \date{} for no date

\pagestyle{empty}

%--------------------

\begin{document}
	

\maketitle % Print the title section

	
\pagebreak


\clearpage


\vspace*{\fill}
\begin{center}
	\begin{minipage}{.75\textwidth}
		
\tableofcontents
\vspace*{0.5cm}
\begin{longtable}[]{@{}lc@{}}
{\bfseries GitHub} & \url{https://github.com/recursion-ninja/CSCI-795-ML} \\
& \\
{\bfseries Video } & \url{https://github.com/recursion-ninja/CSCI-795-ML} \\
\end{longtable}
		
	\end{minipage}
\end{center}
\vfill % equivalent to \vspace{\fill}
\clearpage


\pagebreak


\clearpage
\vspace*{\fill}
\begin{center}
	\begin{minipage}{.9\textwidth}
\hypertarget{abstract}{%
\section{Abstract}\label{abstract}}

In this project, we explore an under-developed aspect of Dungeons and
Dragons 5th Edition (D\&D), the monster "Challenge Rating" system. The
Challenge Rating (CR) is a measure of a monster's lethality against a
party of \emph{four} characters, each that a character level equal to
the monster's CR. Unfortunately, the supplied CR values that have been
published are more often than not poor estimates of monster lethality.
In this project, we attempt to produce an substitute ranking of monster
lethality, ordering them into tiers of either \texttt{{[}1,\ 5{]}},
\texttt{{[}1,\ 10{]}}, or \texttt{{[}1,\ 20{]}}. We hypothesize that the
application of machine learning will produce a better indicator of
monster lethality than the supplied CR score.
	\end{minipage}
\end{center}
\vfill % equivalent to \vspace{\fill}
\clearpage
\pagebreak

\section{Team Member Roles}


{\large\scshape John Lee}

\begin{itemize}
	\item Retrofitted and patched D\&D combat simulator script (too much effort).
	\item Build and evaluated KNN, Logistic Regression, Naive Bayes, and XGBoost models.
	\item Authored project report and constructed presentation slide deck.
\end{itemize}
\vspace*{1em}

{\large\scshape Alex Washburn}

\begin{itemize}
	\item Collected and curated D\&D monster dataset.
	\item Build and evaluated ANN, Decision Tree, Random Forest, and SVM, models.
	\item Authored project report and composed demo video.
\end{itemize}


\section{State-of-the-art}

Some work has been done on applying ML to table top role playing games (TTRPGs) in the past \cite{rameshkumar-bailey-2020-storytelling, macinnes2019d, cavanaugh2016machine, faria2019adaptive, riedl2013interactive}.
Much of the work revolves around the more tractable problem of selecting appropriate ambient music choices for players to experience based on the current emotional tone of the game \cite{ferreira2017mtg, risi2020increasing, padovani2017bardo, ferreira2020computer}.
However, the most popular TTRPG, Dungeons and Dragons (D\&D) has been used as a difficult test-bed for ML experimentation \cite{martin2018dungeons}.
This particular previous work, while quite notable, focused entirely on non-combat aspects of D\&D, eschewing a core component and past time of D\&D; resolving conflict via numerical simulation.
In our project we will grapple with this numeric aspect of D\&D, focusing on a small subset of the D\&D combat system; possession and usage of magical items.
To the best of the author's knowledge, the proposed project will be the first serious attempt to apply machine learning to a numerical aspect of D\&D.
Consequently, there is no \emph{known} previous on which to draw a comparison.

\section{Approach}

\hypertarget{the-dd-monster-data}{%
\subsection{The D\&D Monster Data}\label{the-dd-monster-data}}

We take the stat-block of each monster from the
\href{https://5etools-mirror-1.github.io/}{\texttt{5e.tools}} database
and match the monster with the
\href{https://en.wikipedia.org/wiki/Elo_rating_system}{Elo Ranking} from
\href{https://www.dndcombat.com/dndcombat/Welcome.do?page=Compendium}{\texttt{dndcombat.com}}.
Elo Ranks scores on
\href{https://www.dndcombat.com/dndcombat/Welcome.do?page=Compendium}{\texttt{dndcombat.com}}
are continually updated. The Elo observations were taken on
\texttt{2021-12-08}.

The raw data used for this project has been stored within the repository
for reference:

\begin{itemize}
\tightlist
\item
  \href{https://github.com/recursion-ninja/CSCI-795-ML/tree/main/data/5e.tools}{\texttt{data/5e.tools/*}
  }
\item
  \href{https://github.com/recursion-ninja/CSCI-795-ML/tree/main/data/dndcombat.com}{\texttt{data/dndcombat.com/*}}
\end{itemize}

Both the data from
\href{https://5etools-mirror-1.github.io/}{\texttt{5e.tools}} and
\href{https://www.dndcombat.com/dndcombat/Welcome.do?page=Compendium}{\texttt{dndcombat.com}}
were retrieved in \href{https://en.wikipedia.org/wiki/JSON}{JSON
format}. The data was parsed and curated using a custom tool written by
the authors. The tool is named \texttt{curate-json} and is written in
\href{https://www.haskell.org/}{Haskell}. In order to build the
\texttt{curate-json} tool, the \href{https://www.haskell.org/}{Haskell}
compiler \href{https://www.haskell.org/ghc/download.html}{GHC} and build
tool \href{https://www.haskell.org/cabal/download.html}{\texttt{cabal}}
are required.

The source code for the \texttt{curate-json} tool has been stored within
the repository for reference:

\begin{itemize}
\tightlist
\item
  \href{https://github.com/recursion-ninja/CSCI-795-ML/tree/main/curation}{\texttt{curation/*}}
\end{itemize}

Additionally, there is a
\href{https://github.com/recursion-ninja/CSCI-795-ML/blob/main/curate-dataset.sh}{\texttt{curate-dataset.sh}}
script in the root of the repository which, assuming both
\href{https://www.haskell.org/ghc/download.html}{GHC} and
\href{https://www.haskell.org/cabal/download.html}{\texttt{cabal}} are
installed, will replicate the data curation used for this project. The
\texttt{curate-dataset.sh} script will place the curated dataset in the
\texttt{data} directory.

The curated data used for this project has been stored within the
repository for reference:

\begin{itemize}
\tightlist
\item
  \href{https://github.com/recursion-ninja/CSCI-795-ML/blob/main/data/dnd-5e-monsters.csv}{\texttt{data/dnd-5e-monsters.csv}}
\end{itemize}

\hypertarget{feature-extraction--selection}{%
\subsection{Feature Extraction \& Selection}\label{feature-extraction--selection}}

The curated
\href{https://github.com/recursion-ninja/CSCI-795-ML/blob/main/data/dnd-5e-monsters.csv}{\texttt{dnd-5e-monsters.csv}}
dataset has \texttt{1386} rows and \texttt{72} columns, constituting the
project's initial observations and measurements, respectively. There are
two leading textual columns, labeled
\texttt{\textquotesingle{}Name\textquotesingle{}} and
\texttt{\textquotesingle{}Type\textquotesingle{}} indexed
\texttt{{[}0,\ 1{]}}. The subsequent \texttt{20} columns indexed
\texttt{{[}2,\ 21{]}} are "continuous," integral-valued measurements of
common D\&D attributes. The final
\texttt{\textquotesingle{}Elo\ Rank\textquotesingle{}} column indexed
\texttt{71} is the label for supervised machine learning models. The
next three trailing columns, labeled \texttt{Damage\ Tags},
\texttt{Spellcasting\ Tags} and \texttt{Trait\ Tags} indexed
\texttt{{[}66,\ 70{]}} The remaining columns indexed
\texttt{{[}22,\ 67{]}} are binary indicators for various monster
attributes. A column summary of the initial data set is presented in Table \ref{tab:initial-dataset}.



For our analysis, we desire the
\texttt{\textquotesingle{}Elo\ Rank\textquotesingle{}} to be represented
as an integral "tier list." We extract the tier list feature by
discretizing the 'Elo Rank' column via following procedure:

\begin{enumerate}
\def\labelenumi{\arabic{enumi}.}
\item
  Normalize the data into normally distributed quantiles via the
  \texttt{QuantileTransformer}.
\item
  Decide on the number of tiers: \texttt{{[}1,\ 5{]}},
  \texttt{{[}1,\ 10{]}}, or \texttt{{[}1,\ 20{]}}.
\item
  Scale the data to fit the tier range.
\item
  Remove outliers.
\item
  Round values to nearest integer value.
\end{enumerate}

\begin{table}[!htbp] \centering 
	\caption{The removal of outliers will shrink the dataset depending on tier list size}
	\label{tab:shrunk-observations}
\begin{longtable}[]{@{}rc@{}}
\toprule
Tier List & Observations \\
\midrule
\endhead
\texttt{{[}1,\ \ 5{]}} & \texttt{1200} \\
\texttt{{[}1,\ 10{]}} & \texttt{1316} \\
\texttt{{[}1,\ 20{]}} & \texttt{1348} \\
\bottomrule
\end{longtable}
\end{table}

The final feature extraction involves the last three columns, labeled
\texttt{Damage\ Tags}, \texttt{Spellcasting\ Tags} and
\texttt{Trait\ Tags} indexed \texttt{{[}63,\ 65{]}}. Each can be
expanded to extract an additional features, totaling \texttt{63}
features combined. This feature extraction nearly doubles the fully
expanded dataset size, now totaling to \texttt{135} features.
The fully extracted dataset is presented in Table \ref{tab:fully-extracted-dataset}.

\begin{quote}
{\itshape After extracting all possible features we perform feature selection.}
\end{quote}

First, the textual columns, labeled
\texttt{\textquotesingle{}Name\textquotesingle{}} and
\texttt{\textquotesingle{}Type\textquotesingle{}} indexed
\texttt{{[}0,\ 1{]}} are dropped from our analysis. These columns do not
influence combat efficacy and are not convertible to a meaningful
numeric representation. Their absence makes the machine learning process
much smoother.

Second, we drop all features extracted from the 'Damage Tag' and 'Spell
Tag' columns. These extracted features had essentially no bearing on our
first explored model (Multinomial Naive Bayes), and hence we decided to
omit them from all future models for efficiency reasons.

For the final component of our feature selection process, we perform a "decorrelation" pass through the feature set.
If any pair of features have a correlation coefficient of \texttt{0.6} or higher, we drop one of the columns and use the other as a proxy.
Consequently, we dropped the
features in Table \ref{tab:dropped-features}.
After feature extraction \emph{and} feature selection, our dataset was comprised of \texttt{96} features.
See Table \ref{tab:final-dataset} for the final feature set we used for our machine learning models.


\hypertarget{datset-partitioning}{%
\subsection{Datset partitioning}\label{datset-partitioning}}

We take the prepared dataset and train multiple machine learning
classifiers. We use 80\% of the randomly permuted data as the training
set and the remaining 20\% as the test set. This partition data was
stratified by the \texttt{\textquotesingle{}Elo\ Rank\textquotesingle{}}
column to ensure that each tier is represented. Furthermore, we
partition the training set again, using 80\% as a learning set and the
remaining 20\% as the validation set. Model selection was performed on
the training set; comprised of the "learning" and validation subsets.

\begin{table}[!htbp] \centering 
	\caption{Distribution for partitioning dataset,
		stratified by \texttt{\textquotesingle{}Elo\ Rank\textquotesingle{}}}
	\label{tab:shrunk-observations}
\begin{longtable}[]{@{}cr@{}}
\toprule
Set & Ratio \\
\midrule
\endhead
Test & 20\% \\
Train & 80\% \\
Learn & 64\% \\
Validate & 16\% \\
\bottomrule
\end{longtable}
\end{table}

\hypertarget{model-specification}{%
\subsection{Model Specification}\label{model-specification}}

\begin{enumerate}
\def\labelenumi{\arabic{enumi}.}
\item
  \textbf{Decision Tree:} Decision trees make extremely fast classifiers
  once constructed, but can be incredibly time consuming to build.
  We decided to try our luck with this model and see if an effective
  classifier could be built within a reasonable time frame.
\item
  \textbf{K Nearest Neighbors:} A very simple model with a theoretical bound on it's maximum inaccuracy.
  Because of our familiarity with this model from the extensive discussion in class and it's prominent presence in our first homework, we chose this as our initial classifier to get our bearing and some quick benchmarking numbers.
\item
  \textbf{Logistic Regression:} We read that the a logistic regression
  can be an effective and efficient model for multi-class output, which
  our tier list is. This model was included because we suspected that
  the features had some linear, but not polynomial, relationship(s). The
  logistic regression ought to capture and train well if linear
  relationships exists between the features and the tier list labels.
\item
  \textbf{Multinomial Naive Bayes:} Independent features are am important
  factor for the efficacy of Naive Bayes models. Because we decorrelated
  our dataset, we felt that the remaining correlations beneath the
  \texttt{0.6} threshold was small enough to not interfere with the
  model's performance. Naive Bayes models are supposed to train well on
  small number of observations. Our dataset is just above the
  \texttt{1000} observation threshold, so we had high hopes that this
  model would train well. This was our second model used to get quick
  benchmarking numbers.
\item
  \textbf{Multi-layer Perceptron:} We wanted to experiment with the
  concept of artificial neural nets. The inclusion of this model allowed
  us to to get some experience with an instance of the buzzword-worthy
  model. Given the great flexibility of ANNs, we expected very good
  performance from this model.
\item
  \textbf{Random Forest:} Given the unknown nature and limited domain
  knowledge that we could use to direct the machine learning process,
  the use of at least one ensemble learning technique seemed to be a
  prudent choice. We selected the random forest model for it's ease of
  use and support of multi-output classes.
\item
  \textbf{Support Vector Machines:} The authors have a really solid
  theoretical understanding of how SVMs work so their inclusion was a
  natural choice. Because we have multi-class output, a support vector
  classifier using the "One-versus-One" multi-class strategy strategy was
  used.
\item
  \textbf{X-Gradient Boost:} With the guidance of our professor Anita
  Raja, we were directed to experiment with \texttt{XGBoost} as a
  possibly effective ensemble learning technique which might out perform
  the random forest. This is the model we had the least knowledge of,
  but it served as a great learning opportunity, both from a technical
  and theoretic perspective.
\end{enumerate}

\hypertarget{model-selection}{%
\subsection{Model Selection}\label{model-selection}}

We performed model selection mainly by utilizing the \href{https://scikit-learn.org/stable/modules/generated/sklearn.model_selection.GridSearchCV.html}{\texttt{GridSearchCV}} function.
For each model we considered the smallest multi-class ``tier list'' of \texttt{[0, 5]} and performed a ``wide and sparse search'' over the parameter space.
Based on the results of this initial model selection, we then conducted a more dense hyperparameter search, centered about the results from the initial sparse search.
This two pass approach proved quite effective both in terms of runtime efficiency and classifier efficacy.

After model selection was complete for the multi-class label range \texttt{[0, 5]}, we applied the same hyperparameters to the larger ``tier lists'' with multi-class label ranges \texttt{[0, 10]} and \texttt{[0, 20]}.
Unsurprisingly, since the models were not tuned for these larger multi-class labels ranges, their performance scores suffered significantly.
We attempted to begin model selection for a the larger ranges, but this was not possible given the time constraints of the semester and our initial setbacks described below in the subsection \textbf{Challenges Faced}.
The hyperparameters resulting from our model selection process can be found by referencing the executable file corresponding to the model listed in Table \ref{tab:hyperparameters}, and inspecting the definition of \texttt{hyperparameter\_values} within the file.


\section{Evaluation}

\section{Discussion}

\subsection{Challenges Faced}
\label{sec:challenges}

Originally, the plan was to do a more in-depth analysis of magical items. This hit a snag when great difficulty was found with the automated combat simulator being used for the dataset generation. In particular, the libraries had many conflicts and missing sections. The uploader had accidentally pushed an experimental version, which called upon new functions not found in the original. Additionally, an entire section of the simulation would have remained broken, greatly reducing the accuracy of the predictions in regards to creatures and classes with spell-casting capability. We scaled this back even further, attempting to simulate basic combat between brawlers and simple creatures, only to get heavily skewed results that were highly questionable, such as a greater than 70\% loss rate for aboleths against itself out of 10000 runs. Similarly, team combat ended the same way, rendering even a simple simulation of basic mechanics completely useless.
\newline
Eventually, it was decided to use a dataset that already had computed values based on effectiveness. This use of "Elo rating system" is based on the same one used for competitive zero-sum games, and is relatively well understood. The source used their own \href{https://www.dndcombat.com/dndcombat/Welcome.do}{\texttt{AI-based combat simulator}} with working class features to generate the ELO ratings, based off of survival chances and victories scored by \href{https://www.dndcombat.com/code_updates.html?v=3.51}{textt{millions of simulated fights}} run by various individuals. Although the original goal had to be changed, it was done quickly with little down-time.
\newline


\subsection{Lessons Learned}

\clearpage

\section{Bibliography}

\bibliographystyle{IEEEtran}
\bibliography{report}

\clearpage

\section{Appendix}

\begin{table}[!htbp] \centering 
	\caption{\bfseries Initial Dataset} 
	\label{tab:initial-dataset}

\begin{footnotesize}
\begin{minipage}[b]{0.45\linewidth}\centering
\begin{longtable}[]{@{}rll@{}}
	\toprule
	\# & Column Label & Data Type \\
	\midrule
	\endhead
	0 & Name & Textual \\
	1 & Type & Textual \\
	2 & Size & Integral \\
	3 & Armor & Integral \\
	4 & Hit Points & Integral \\
	5 & Move Burrow & Integral \\
	6 & Move Climb & Integral \\
	7 & Move Fly & Integral \\
	8 & Move Swim & Integral \\
	9 & Move Walk & Integral \\
	10 & Stat Str & Integral \\
	11 & Stat Dex & Integral \\
	12 & Stat Con & Integral \\
	13 & Stat Int & Integral \\
	14 & Stat Wis & Integral \\
	15 & Stat Cha & Integral \\
	16 & Save Str & Integral \\
	17 & Save Dex & Integral \\
	18 & Save Con & Integral \\
	19 & Save Int & Integral \\
	20 & Save Wis & Integral \\
	21 & Save Cha & Integral \\
	22 & Blind Sight & Binary \\
	23 & Dark Vision & Binary \\
	24 & Tremorsense & Binary \\
	25 & True Sight & Binary \\
	26 & Immune Acid & Binary \\
	27 & Immune Bludgeoning & Binary \\
	28 & Immune Cold & Binary \\
	29 & Immune Fire & Binary \\
	30 & Immune Force & Binary \\
	31 & Immune Lightning & Binary \\
	32 & Immune Necrotic & Binary \\
	33 & Immune Piercing & Binary \\
	34 & Immune Poison & Binary \\
	35 & Immune Psychic & Binary \\
	\bottomrule
\end{longtable}
\end{minipage}
\hspace{0.5cm}
\begin{minipage}[b]{0.45\linewidth}\centering
\begin{longtable}{@{}rll@{}}
	\toprule
	\# & Column Label & Data Type \\
	\midrule
	\endhead
	36 & Immune Radiant & Binary \\
	37 & Immune Slashing & Binary \\
	38 & Immune Thunder & Binary \\
	39 & Resist Acid & Binary \\
	40 & Resist Bludgeoning & Binary \\
	41 & Resist Cold & Binary \\
	42 & Resist Fire & Binary \\
	43 & Resist Force & Binary \\
	44 & Resist Lightning & Binary \\
	45 & Resist Necrotic & Binary \\
	46 & Resist Piercing & Binary \\
	47 & Resist Poison & Binary \\
	48 & Resist Psychic & Binary \\
	49 & Resist Radiant & Binary \\
	50 & Resist Slashing & Binary \\
	51 & Resist Thunder & Binary \\
	52 & Cause Blinded & Binary \\
	53 & Cause Charmed & Binary \\
	54 & Cause Deafened & Binary \\
	55 & Cause Frightened & Binary \\
	56 & Cause Grappled & Binary \\
	57 & Cause Incapacitated & Binary \\
	58 & Cause Invisible & Binary \\
	59 & Cause Paralyzed & Binary \\
	60 & Cause Petrified & Binary \\
	61 & Cause Poisoned & Binary \\
	62 & Cause Prone & Binary \\
	63 & Cause Restrained & Binary \\
	64 & Cause Stunned & Binary \\
	65 & Cause Unconscious & Binary \\
	66 & Multiattack & Binary \\
	67 & Spellcasting & Binary \\
	68 & Damage Tags & Textual \\
	69 & Spellcasting Tags & Textual \\
	70 & Trait Tags & Textual \\\title{\texttt{\LARGE{Predicting Magical Item Effectiveness in \\Dungeons and Dragons 5th Edition} \\\vspace{-0.35cm} {\large A Hunter College CSCI-795 Project Proposal}\\\normalsize\url{https://github.com/recursion-ninja/CSCI-795-ML}}} % Title and subtitle
	71 & Elo Rank & Integral \\
	\bottomrule
\end{longtable}
\end{minipage}
\end{footnotesize}

\end{table}

\begin{table}[ht]
	\caption{\bfseries Fully Extracted Dataset}
	\label{tab:fully-extracted-dataset}
\begin{tiny}
\begin{minipage}[b]{0.45\linewidth}\centering
\begin{tabular}{@{}rll@{}}
	\toprule
	\# & Column Label & Data Type \\
	\midrule
	0 & Name & Textual \\
	1 & Type & Textual \\
	2 & Size & Integral \\
	3 & Armor & Integral \\
	4 & Hit Points & Integral \\
	5 & Move Burrow & Integral \\
	6 & Move Climb & Integral \\
	7 & Move Fly & Integral \\
	8 & Move Swim & Integral \\
	9 & Move Walk & Integral \\
	10 & Stat Str & Integral \\
	11 & Stat Dex & Integral \\
	12 & Stat Con & Integral \\
	13 & Stat Int & Integral \\
	14 & Stat Wis & Integral \\
	15 & Stat Cha & Integral \\
	16 & Save Str & Integral \\
	17 & Save Dex & Integral \\
	18 & Save Con & Integral \\
	19 & Save Int & Integral \\
	20 & Save Wis & Integral \\
	21 & Save Cha & Integral \\
	22 & Blind Sight & Binary \\
	23 & Dark Vision & Binary \\
	24 & Tremorsense & Binary \\
	25 & True Sight & Binary \\
	26 & Immune Acid & Binary \\
	27 & Immune Bludgeoning & Binary \\
	28 & Immune Cold & Binary \\
	29 & Immune Fire & Binary \\
	30 & Immune Force & Binary \\
	31 & Immune Lightning & Binary \\
	32 & Immune Necrotic & Binary \\
	33 & Immune Piercing & Binary \\
	34 & Immune Poison & Binary \\
	35 & Immune Psychic & Binary \\
	36 & Immune Radiant & Binary \\
	37 & Immune Slashing & Binary \\
	38 & Immune Thunder & Binary \\
	39 & Resist Acid & Binary \\
	40 & Resist Bludgeoning & Binary \\
	41 & Resist Cold & Binary \\
	42 & Resist Fire & Binary \\
	43 & Resist Force & Binary \\
	44 & Resist Lightning & Binary \\
	45 & Resist Necrotic & Binary \\
	46 & Resist Piercing & Binary \\
	47 & Resist Poison & Binary \\
	48 & Resist Psychic & Binary \\
	49 & Resist Radiant & Binary \\
	50 & Resist Slashing & Binary \\
	51 & Resist Thunder & Binary \\
	52 & Cause Blinded & Binary \\
	53 & Cause Charmed & Binary \\
	54 & Cause Deafened & Binary \\
	55 & Cause Frightened & Binary \\
	56 & Cause Grappled & Binary \\
	57 & Cause Incapacitated & Binary \\
	58 & Cause Invisible & Binary \\
	59 & Cause Paralyzed & Binary \\
	60 & Cause Petrified & Binary \\
	61 & Cause Poisoned & Binary \\
	62 & Cause Prone & Binary \\
	63 & Cause Restrained & Binary \\
	64 & Cause Stunned & Binary \\
	65 & Cause Unconscious & Binary \\
	66 & Multiattack & Binary \\
	67 & Spellcasting & Binary \\
	\bottomrule
\end{tabular}
\end{minipage}
\hspace{0.5cm}
\begin{minipage}[b]{0.45\linewidth}\centering
\begin{tabular}{@{}rll@{}}
	\toprule
	\# & Column Label & Data Type \\
	\midrule
	68 & Damage\_A & Binary \\
	69 & Damage\_B & Binary \\
	70 & Damage\_C & Binary \\
	71 & Damage\_F & Binary \\
	72 & Damage\_I & Binary \\
	73 & Damage\_L & Binary \\
	74 & Damage\_N & Binary \\
	75 & Damage\_O & Binary \\
	76 & Damage\_P & Binary \\
	77 & Damage\_R & Binary \\
	78 & Damage\_S & Binary \\
	79 & Damage\_T & Binary \\
	80 & Damage\_Y & Binary \\
	81 & Spellcasting\_CA & Binary \\
	82 & Spellcasting\_CB & Binary \\
	83 & Spellcasting\_CC & Binary \\
	84 & Spellcasting\_CD & Binary \\
	85 & Spellcasting\_CL & Binary \\
	86 & Spellcasting\_CP & Binary \\
	87 & Spellcasting\_CR & Binary \\
	88 & Spellcasting\_CS & Binary \\
	89 & Spellcasting\_CW & Binary \\
	90 & Spellcasting\_F & Binary \\
	91 & Spellcasting\_I & Binary \\
	92 & Spellcasting\_P & Binary \\
	93 & Spellcasting\_S & Binary \\
	94 & Aggressive & Binary \\
	95 & Ambusher & Binary \\
	96 & Amorphous & Binary \\
	97 & Amphibious & Binary \\
	98 & Antimagic Susceptibility & Binary \\
	99 & Brute & Binary \\
	100 & Charge & Binary \\
	101 & Damage Absorption & Binary \\
	102 & Death Burst & Binary \\
	103 & Devil's Sight & Binary \\
	104 & False Appearance & Binary \\
	105 & Fey Ancestry & Binary \\
	106 & Flyby & Binary \\
	107 & Hold Breath & Binary \\
	108 & Illumination & Binary \\
	109 & Immutable Form & Binary \\
	110 & Incorporeal Movement & Binary \\
	111 & Keen Senses & Binary \\
	112 & Legendary Resistances & Binary \\
	113 & Light Sensitivity & Binary \\
	114 & Magic Resistance & Binary \\
	115 & Magic Weapons & Binary \\
	116 & Pack Tactics & Binary \\
	117 & Pounce & Binary \\
	118 & Rampage & Binary \\
	119 & Reckless & Binary \\
	120 & Regeneration & Binary \\
	121 & Rejuvenation & Binary \\
	122 & Shapechanger & Binary \\
	123 & Siege Monster & Binary \\
	124 & Sneak Attack & Binary \\
	125 & Spell Immunity & Binary \\
	126 & Spider Climb & Binary \\
	127 & Sunlight Sensitivity & Binary \\
	128 & Turn Immunity & Binary \\
	129 & Turn Resistance & Binary \\
	130 & Undead Fortitude & Binary \\
	131 & Water Breathing & Binary \\
	132 & Web Sense & Binary \\
	133 & Web Walker & Binary \\
	134 & Elo Rank & Integral \\
	\bottomrule
\end{tabular}
\end{minipage}
\end{tiny}

\end{table}

\begin{table}[!htbp] \centering 
	\caption{Dropped highly correlated features and their proxy}
	\label{tab:dropped-features}
	
\begin{longtable}[]{@{}ll@{}}
	\toprule
	Proxy & Dropped Feature \\
	\midrule
	\endhead
	Hit Points & Size \\
	Hit Points & Stat Str \\
	Hit Points & Stat Con \\
	Stat Cha & Stat Int \\
	Resist Fire & Resist Cold \\
	Resist Fire & Resist Lightning \\
	Resist Acid & Resist Thunder \\
	Resist Acid & Incorporeal Movement \\
	Damage Absorption & Immutable Form \\
	Spell Immunity & Immune Force \\
	Web Sense & Web Walker \\
	\bottomrule
\end{longtable}

\end{table}


\begin{table}[!htbp] \centering 
	\caption{\bfseries Final Dataset}
	\label{tab:final-dataset}
\begin{scriptsize}
\begin{minipage}[b]{0.45\linewidth}\centering
\begin{longtable}[]{@{}rll@{}}
	\toprule
	\# & Column Label & Data Type \\
	\midrule
	\endhead
	0 & Armor & Integral \\
	1 & Hit Points & Integral \\
	2 & Move Burrow & Integral \\
	3 & Move Climb & Integral \\
	4 & Move Fly & Integral \\
	5 & Move Swim & Integral \\
	6 & Move Walk & Integral \\
	7 & Stat Dex & Integral \\
	8 & Stat Wis & Integral \\
	9 & Stat Cha & Integral \\
	10 & Save Str & Integral \\
	11 & Save Dex & Integral \\
	12 & Save Con & Integral \\
	13 & Save Int & Integral \\
	14 & Save Wis & Integral \\
	15 & Save Cha & Integral \\
	16 & Blind Sight & Binary \\
	17 & Dark Vision & Binary \\
	18 & Tremorsense & Binary \\
	19 & True Sight & Binary \\
	20 & Immune Acid & Binary \\
	21 & Immune Bludgeoning & Binary \\
	22 & Immune Cold & Binary \\
	23 & Immune Fire & Binary \\
	24 & Immune Lightning & Binary \\
	25 & Immune Necrotic & Binary \\
	26 & Immune Piercing & Binary \\
	27 & Immune Poison & Binary \\
	28 & Immune Psychic & Binary \\
	29 & Immune Radiant & Binary \\
	30 & Immune Slashing & Binary \\
	31 & Immune Thunder & Binary \\
	32 & Resist Bludgeoning & Binary \\
	33 & Resist Cold & Binary \\
	34 & Resist Force & Binary \\
	35 & Resist Necrotic & Binary \\
	36 & Resist Piercing & Binary \\
	37 & Resist Poison & Binary \\
	38 & Resist Psychic & Binary \\
	39 & Resist Radiant & Binary \\
	40 & Resist Slashing & Binary \\
	41 & Cause Blinded & Binary \\
	42 & Cause Charmed & Binary \\
	43 & Cause Deafened & Binary \\
	44 & Cause Frightened & Binary \\
	45 & Cause Grappled & Binary \\
	46 & Cause Incapacitated & Binary \\
	47 & Cause Invisible & Binary \\
	\bottomrule
\end{longtable}
\end{minipage}
\hspace{0.5cm}
\begin{minipage}[b]{0.45\linewidth}\centering
\begin{longtable}{@{}rll@{}}
	\toprule
	\# & Column Label & Data Type \\
	\midrule
	\endhead
	48 & Cause Paralyzed & Binary \\
	49 & Cause Petrified & Binary \\
	50 & Cause Poisoned & Binary \\
	51 & Cause Prone & Binary \\
	52 & Cause Restrained & Binary \\
	53 & Cause Stunned & Binary \\
	54 & Cause Unconscious & Binary \\
	55 & Multiattack & Binary \\
	56 & Spellcasting & Binary \\
	57 & Aggressive & Binary \\
	58 & Ambusher & Binary \\
	59 & Amorphous & Binary \\
	60 & Amphibious & Binary \\
	61 & Antimagic Susceptibility & Binary \\
	62 & Brute & Binary \\
	63 & Charge & Binary \\
	64 & Death Burst & Binary \\
	65 & Devil's Sight & Binary \\
	66 & False Appearance & Binary \\
	67 & Fey Ancestry & Binary \\
	68 & Flyby & Binary \\
	69 & Hold Breath & Binary \\
	70 & Illumination & Binary \\
	71 & Immutable Form & Binary \\
	72 & Incorporeal Movement & Binary \\
	73 & Keen Senses & Binary \\
	74 & Legendary Resistances & Binary \\
	75 & Light Sensitivity & Binary \\
	76 & Magic Resistance & Binary \\
	77 & Magic Weapons & Binary \\
	78 & Pack Tactics & Binary \\
	79 & Pounce & Binary \\
	80 & Rampage & Binary \\
	81 & Reckless & Binary \\
	82 & Regeneration & Binary \\
	83 & Rejuvenation & Binary \\
	84 & Shapechanger & Binary \\
	85 & Siege Monster & Binary \\
	86 & Sneak Attack & Binary \\
	87 & Spell Immunity & Binary \\
	88 & Spider Climb & Binary \\
	89 & Sunlight Sensitivity & Binary \\
	90 & Turn Immunity & Binary \\
	91 & Turn Resistance & Binary \\
	92 & Undead Fortitude & Binary \\
	93 & Water Breathing & Binary \\
	94 & Web Walker & Binary \\
	95 & Elo Rank & Integral \\
	\bottomrule
\end{longtable}
\end{minipage}
\end{scriptsize}

\end{table}


\begin{table}[!htbp] \centering 
	\caption{ Model \& file containing corresponding \texttt{hyperparameter\_values} definition.}
	\label{tab:hyperparameters}
\begin{longtable}[]{@{}ll@{}}
	\toprule
	Model Name & Executable File \\
	\midrule
	\endhead
	Decision Tree &
	\href{https://github.com/recursion-ninja/CSCI-795-ML/blob/main/classification/model_DecisionTree.py}{\texttt{classification/model\_DecisionTree.py}
	} \\
	K Nearest Neighbors &
	\href{https://github.com/recursion-ninja/CSCI-795-ML/blob/main/classification/model_KNN.py}{\texttt{classification/model\_KNN.py}
	} \\
	Logistic Regression &
	\href{https://github.com/recursion-ninja/CSCI-795-ML/blob/main/classification/model_LogisticRegression.py}{\texttt{classification/model\_LogisticRegression.py}} \\
	Multinomial Naive Bayes &
	\href{https://github.com/recursion-ninja/CSCI-795-ML/blob/main/classification/model_NaiveBayes.py}{\texttt{classification/model\_NaiveBayes.py}
	} \\
	Multi-layer Perceptron &
	\href{https://github.com/recursion-ninja/CSCI-795-ML/blob/main/classification/model_NeuralNetwork.py}{\texttt{classification/model\_NeuralNetwork.py}
	} \\
	Random Forest &
	\href{https://github.com/recursion-ninja/CSCI-795-ML/blob/main/classification/model_RandomForest.py}{\texttt{classification/model\_RandomForest.py}
	} \\
	Support Vector Machines &
	\href{https://github.com/recursion-ninja/CSCI-795-ML/blob/main/classification/model_SVM.py}{\texttt{classification/model\_SVM.py}
	} \\
	X-Gradiant Boost &
	\href{https://github.com/recursion-ninja/CSCI-795-ML/blob/main/classification/model_XGBoost.py}{\texttt{classification/model\_XGBoost.py}
	} \\
	\bottomrule
\end{longtable}
\end{table}


\end{document}
